\documentclass[11pt]{scrartcl}


 
\usepackage{geometry}

\geometry{verbose,letterpaper,tmargin=1in,bmargin=1in,lmargin=1in,rmargin=1in}


 \usepackage{xfrac} 

\usepackage{amsmath} 

\usepackage{amssymb} 

\usepackage{bussproofs}

 \usepackage{fontspec} 
 
 \usepackage{xunicode} 
 
 \usepackage{xltxtra} 
 
 \defaultfontfeatures{Mapping=tex-text} 
 
 %\setmainfont{Thorndale AMT}
 

 \setlength{\parindent}{0em}
 
 \setlength{\parskip}{1em}
 
 \begin{document}

 \section*{Systems of Modal Logic}



All systems of modal logic are extensions of propositional logic. They accept a
common inference rule but differ by their characteristic axioms (more
precisely, axiom schemata).  To formulate the inference rule and axioms, we
introduce the following primitive operator: 

$\Box\phi$: ``it is necessarily true that $\phi$''.

It is convenient to also introduce the following operator:

$\Diamond\phi \Longleftrightarrow \neg\Box\neg\phi$: ``it is possibly true that
$\phi$'' ($\phi$ is not necessarily false iff.  it is possibly true).

Now to the modal systems. The common inference rule is:

\begin{description}

 \item[Necessitation] If $\phi$ is a truth of logic, infer $\Box\phi$.

  \hspace{5em} \AxiomC{$\phi$} \UnaryInfC{$\Box\phi$}\DisplayProof


\end{description}


\subsection*{The Lewis Modal Systems}

\begin{description}

  \setlength{\itemsep}{2em}

\item[System K]

Characteristic Axiom: $\Box(\phi\supset\psi)\supset(\Box\phi\supset\Box\psi)$

If it is necessarily true that if $\phi$ then $\psi$, then: if it is
necessarily true that $\phi$, then it is also necessarily true that $\psi$. 

\item[System D]

System K plus

Characteristic Axiom: $\Box\phi\supset\Diamond\phi$

If $\phi$ is necessarily true, then it is possibly true.

\item[System T]

System K plus

Characteristic Axiom: $\Box\phi\supset\phi$

If it is necessarily true that $\phi$, then it is actually true that $\phi$. 

\item[System B]

System T plus

Characteristic Axiom: $\Diamond\Box\phi\supset\phi$

If $\phi$ is possibly necessarily true, then $\phi$ is actually true.


\item[System S4]

System T plus

Characteristic Axiom: $\Box\phi\supset\Box\Box\phi$

Whatever is necessarily true is necessarily necessary. 

\item[System S5]

System T plus

Characteristic Axiom: $\Diamond\Box\phi\supset\Box\phi$

Whatever is possibly necessary is necessary.


\end{description}

\vskip 4em

\section*{Exercises}

\begin{enumerate}

 \item Show that the following are provable in all modal systems:

  \begin{itemize}

   \item $\Box(p\wedge q)\supset(\Box p \supset \Box q)$ 

   \item $(\Box p \vee \Box q) \supset \Box(p \vee q)$

   \item $\Box(p \supset q) \supset (\Diamond p \supset \Diamond q)$

  \end{itemize}

 \item For each of the formulae in the previous question, explain in plain
  English (no formal symbols!) why they are plausible.

 \item Prove that the characteristic axioms of B and S4 are theorems of S5.

 \item An alternative reading of $\Box\phi$ is: it will always be the case that
  $\phi$. 
  
  \begin{enumerate} 
   
  \item On this reading, what would $\Diamond\phi$ mean? 

  \item How plausible or implausible are the various modal systems under this
   interpretation? Explain for each system.

 \end{enumerate}

\item Here is yet another reading of the modal symbols. Let $\Diamond p$ mean
 that the truth of $p$ is consistent with your knowledge (roughly, for all you
 know, $p$ could be true). 

 \begin{enumerate}

  \item On this reading, what would $\Box\phi$ mean?

  \item  How plausible or implausible are the various modal systems under this
   interpretation? Explain for each system.

 \end{enumerate}

\end{enumerate}
  
  
  \end{document}

